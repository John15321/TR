\documentclass{article}
\usepackage{amsmath}
\usepackage{amssymb}
\usepackage{mathtools}
\begin{document}

\title{Lista 2, zadania 3, 4, 5}
\author{Jan Bronicki 249011}
\date{}
\maketitle

\section*{Zadanie 3}
% [label=\alph*)]
\begin{enumerate}

    \item[a)] $K(s)=\frac{1}{s(s + 1)}$

    Odpowiedź skokowa:
    
    $$ \lambda(t)=\mathcal{L}^{-1}\left\{\frac{1}{s}\cdot K(s)\right\} $$
    
    $$ \lambda(t)=\mathcal{L}^{-1}\left\{  \frac{1}{s \left(s^{2} + s\right)} \right\}$$
    
    $$ \lambda(t)=\mathcal{L}^{-1}\left\{  \frac{1}{s + 1} - \frac{1}{s} + \frac{1}{s^{2}} \right\}$$
    
    $$ \lambda(t) = t - 1 + e^{- t} $$
    
    $$ \lambda^{\prime}(t) = 1 - e^{- t} $$
    
    $$ \lambda^{\prime\prime}(t) = e^{- t} $$
    
    $$ \lambda(0) =  0$$
    
    $$ \lambda^{\prime}(0) = 0$$
    
    $$ \lambda^{\prime\prime}(0) = 1 $$
    
    \newpage
    
    Odpowiedź impulsowa:
    
    $$ k(t)=\mathcal{L}^{-1}\left\{K(s)\right\} $$
    
    $$ k(t)=\mathcal{L}^{-1}\left\{ \frac{1}{s^{2} + s} \right\} $$
    
    $$ k(t)=\mathcal{L}^{-1}\left\{  - \frac{1}{s + 1} + \frac{1}{s}  \right\} $$
    
    $$ k(t) =  1 - e^{- t} $$
    
    $$ k^{\prime}(t) = e^{- t} $$
    
    $$ k^{\prime\prime}(t) = - e^{- t} $$
    
    $$ k(0) =  0 $$
    
    $$ k^{\prime}(0) = 1 $$
    
    $$ k^{\prime\prime}(0) = - 1 $$


    \newpage

    
    
    
    
    \item[b)]  $K(s)=\frac{1}{\left(s + 1\right) \left(s^{2} + 1\right)}$

    Odpowiedź skokowa:

    $$ \lambda(t)=\mathcal{L}^{-1}\left\{\frac{1}{s}\cdot K(s)\right\} $$

    $$ \lambda(t)=\mathcal{L}^{-1}\left\{\frac{1}{s}\cdot \frac{1}{\left(s + 1\right) \left(s^{2} + 1\right)}\right\} $$

    $$ \lambda(t)= \mathcal{L}^{-1}\left\{- \frac{s + 1}{2 \left(s^{2} + 1\right)} - \frac{1}{2 \left(s + 1\right)} + \frac{1}{s}
    \right\}$$

    $$ \lambda(t)=- \frac{\sin{\left(t \right)}}{2} - \frac{\cos{\left(t \right)}}{2} + 1 - \frac{e^{- t}}{2} $$

    $$ \lambda^{\prime}(t)=\frac{\sin{\left(t \right)}}{2} - \frac{\cos{\left(t \right)}}{2} + \frac{e^{- t}}{2}$$

    $$ \lambda^{\prime\prime}(t)=\frac{\sin{\left(t \right)}}{2} + \frac{\cos{\left(t \right)}}{2} - \frac{e^{- t}}{2}$$

    $$ \lambda(0)= 0 $$

    $$ \lambda^{\prime}(0)= 0 $$

    $$ \lambda^{\prime\prime}(0)= 0 $$

    \newpage


    Odpowiedź impulsowa:

    $$ k(t)=\mathcal{L}^{-1}\left\{K(s)\right\} $$

    $$ k(t)=\mathcal{L}^{-1}\left\{\frac{1}{\left(s + 1\right) \left(s^{2} + 1\right)}\right\} $$

    $$ k(t)=\mathcal{L}^{-1}\left\{- \frac{s - 1}{2 \left(s^{2} + 1\right)} + \frac{1}{2 \left(s + 1\right)}\right\} $$

    $$ k(t)=\frac{\sin{\left(t \right)}}{2} - \frac{\cos{\left(t \right)}}{2} + \frac{e^{- t}}{2}$$

    $$ k^{\prime}(t)=\frac{\sin{\left(t \right)}}{2} + \frac{\cos{\left(t \right)}}{2} - \frac{e^{- t}}{2}$$
    
    $$ k^{\prime\prime}(t)=- \frac{\sin{\left(t \right)}}{2} + \frac{\cos{\left(t \right)}}{2} + \frac{e^{- t}}{2}$$

    $$ k(0)=0$$

    $$ k^{\prime}(0)=0$$

    $$ k^{\prime\prime}(0)=1$$

    \newpage


    \item[c)]  $K(s)=\frac{1}{\left(s - 1\right) \left(s + 2\right)}$

    Odpowiedź skokowa:

    $$ \lambda(t)=\mathcal{L}^{-1}\left\{\frac{1}{s}\cdot K(s)\right\} $$

    $$ \lambda(t)=\mathcal{L}^{-1}\left\{\frac{1}{s \left(s - 1\right) \left(s + 2\right)}\right\} $$

    $$ \lambda(t)=\mathcal{L}^{-1}\left\{\frac{1}{6 \left(s + 2\right)} + \frac{1}{3 \left(s - 1\right)} - \frac{1}{2 s}\right\}$$

    $$ \lambda(t)=\frac{e^{t}}{3} - \frac{1}{2} + \frac{e^{- 2 t}}{6}$$

    $$ \lambda^{\prime}(t)=\frac{e^{t}}{3} - \frac{e^{- 2 t}}{3}$$

    $$ \lambda^{\prime\prime}(t)=\frac{e^{t}}{3} + \frac{2 e^{- 2 t}}{3}$$

    $$ \lambda(0)=0 $$

    $$ \lambda^{\prime}(0)=0$$

    $$ \lambda^{\prime\prime}(0)=1$$

    \newpage

    Odpowiedź impulsowa:

    $$ k(t)=\mathcal{L}^{-1}\left\{K(s)\right\} $$

    $$ k(t)=\mathcal{L}^{-1}\left\{\frac{1}{\left(s - 1\right) \left(s + 2\right)}\right\} $$

    $$ k(t)=\mathcal{L}^{-1}\left\{- \frac{1}{3 \left(s + 2\right)} + \frac{1}{3 \left(s - 1\right)}\right\} $$

    $$ k(t)=\frac{e^{t}}{3} - \frac{e^{- 2 t}}{3}$$

    $$ k^{\prime}(t)=\frac{e^{t}}{3} + \frac{2 e^{- 2 t}}{3}$$

    $$ k^{\prime\prime}(t)=\frac{e^{t}}{3} - \frac{4 e^{- 2 t}}{3}$$

    $$ k(0)=0$$

    $$ k^{\prime}(0)=1$$

    $$ k^{\prime\prime}(0)=-1$$
\end{enumerate}

\newpage


\section*{Zadanie 4}

\begin{enumerate}
    \item[a)] 
    $$ K(s) =  \frac{1}{(s+1)(s+3)} = \frac{L(s)}{M(s)} $$

    $$ Y(s) = K(s)U(s) = \frac{L(s)}{M(s)}U(s) $$

    $$Y(s)\left[s^{2}+4s+3\right]=[s+2]U(s)$$

    $$ y^{\prime\prime}+4y^{\prime}+3y=u^{\prime}+2u $$ 


    \item[b)] 

    $$ y^{\prime\prime}+4y^{\prime}+3y = 0 $$

    $$ y^{\prime\prime} = -4y^{\prime} - 3y $$

    $$ \dot{\xi} = A\xi$$

    $$ \xi = \begin{bmatrix}
        y \\
        y^{\prime}
        \end{bmatrix} $$

    $$ \dot{\xi} =  \begin{bmatrix}
        y^{\prime} \\
        y^{\prime\prime}
        \end{bmatrix}$$


    $$ \begin{bmatrix}
        y^{\prime} \\
        y^{\prime\prime}
        \end{bmatrix} = \begin{bmatrix}
            0 & 1\\
            -3 & -4
            \end{bmatrix}
        \;
        \begin{bmatrix}
            y \\
            y^{\prime}
            \end{bmatrix}   
             $$
    
\end{enumerate}

\newpage

\section*{Zadanie 5}


\subsection*{a)}
Drugi biegun jest sprzezony, dlatego wynosi on:
$$s_{2}=\sigma - j\omega$$

\subsection*{b)}

$$ K(s)=\frac{1}{M(s)}=\frac{1}{(s-s_{1})(s-s_{2})}=\frac{1}{s^{2}-2s\delta+\delta +\omega^{2}} $$
\subsection*{$\alpha)$}

$$ \omega=0, \delta>0 $$

$$ \frac{1}{s^{2}+\delta^{2}-2s\delta}=\frac{1}{s-\delta}=e^{\delta t}\cdot t\cdot 1 $$
\subsection*{$\beta)$}
$$\omega=0, \delta<0$$

$$ \frac{1}{(s+\delta)^{2}}=\frac{1}{(s+\delta)^{2}}=e^{-\delta t}\cdot t \cdot 1(t) $$

\subsection*{$\gamma)$}

$$\omega\neq 0, \delta>0$$

$$ \frac{1}{s^{2}-2s\delta + \delta + \omega^{2}} = \frac{1}{\omega}\cdot \frac{\omega}{(s-\delta)^{2}+\omega^{2}}$$

$$ y(t)=\frac{1}{\omega}e^{\delta t}\cdot sin(\omega t)\cdot 1(t) $$


\subsection*{$\delta)$}

$$ \omega \neq 0, \delta<0$$

$$ \frac{1}{(s+\delta)^{2}+\omega^{2}}= \frac{1}{\omega}\frac{\omega}{(s+\delta)^{2}+\omega^{2}} $$

$$ y(t) = \frac{1}{\omega}\cdot e^{-\delta t} sin(\omega t) \cdot 1(t)$$



\end{document}