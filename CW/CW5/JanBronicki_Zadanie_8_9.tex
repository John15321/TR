\documentclass{article}
\usepackage{polski}
\usepackage[utf8]{inputenc}


\title{Teoria Regulacji Ćwiczenia, Wtorek 17:05-18:45}
\author{Jan Bronicki 249011 }
\date{}

\begin{document}

\maketitle


\begin{center}
    Lista 2
\end{center}

\section*{Zadanie 8}

Oryginalna transmitancja:

$$ K(s) = \frac{1}{\left(s + 1\right) \left(s + 2\right)} $$

Podany sygnał:

$$ u(t) = 4sin(\omega t) $$

Transmitancja widmowa:

$$ s = i\omega $$

$$K(i\omega) = \frac{1}{\left(i w + 1\right) \left(i w + 2\right)}$$

Dążymy do postaci wykładniczej:

$$ Z=r\cdot e^{i\phi} $$

$$ r = |Z| $$

$$ \phi = \tan^{-1}\left(\frac{Im}{Re}\right) $$

Postać sygnału na wyjściu:

$$y(t)\approx\left|K(i\omega)\right|sin\left(\omega t + argK(i \omega)\right)$$



$$ \left|K(i\omega)\right| = \sqrt{\frac{1}{(\omega^{2}+1)(\omega^{2}+4)}}\cdot e^{j\cdot 
tan^{-1}\left(\frac{-3\omega}{2-\omega^{2}}\right)} $$

Podstawiam do wzoru:

$$y(t)=4\cdot \sqrt{\frac{1}{754}}\cdot \sin\left(5t+0.577\right)$$












\newpage

\section*{Zadanie 9}

$$ \lim\limits_{t \to \infty} \lambda(t) = K(0) = \lim\limits_{s \to 0} K(s)  $$

\begin{itemize}
    \item[a)] $\frac{s+2}{(s+3)(s+4)^{2}}$
     
        $$\lim\limits_{t \to \infty} \lambda(t)=\lim\limits_{s \to 0} K(s)=
        \lim\limits_{s \to 0} \frac{s+2}{(s+3)(s+4)^{2}}=\frac{1}{24}$$

        $$ s_{1}=-3, \ s_{2}=-4, \ s_{3}=-4 $$

        Wsp. wzmocnienia istnieje, system jest stabilny.

    \item[b)] $\frac{3}{(s-1)(s+2)}$
    
        $$\lim\limits_{t \to \infty} \lambda(t)=\lim\limits_{s \to 0} K(s)=
        \lim\limits_{s \to 0} \frac{3}{(s-1)(s+2)}$$

        $$ s_{1}=1, \ s_{2}=-2$$
        
        Niestabilne, wzmocnienie nie istnieje.

    \item[c)] $ \frac{4}{s(s+6)} $
    
        $$\lim\limits_{t \to \infty} \lambda(t)=\lim\limits_{s \to 0} K(s)=
        \lim\limits_{s \to 0} \frac{4}{s(s+6)}$$

        $$ s_{1}=0, \ s_{2}=-6 $$

        Na granicy stabilności, wzmocnienie nie istnieje.

    \item[d)] $\frac{1}{s^{2}+1}$  

        $$\lim\limits_{t \to \infty} \lambda(t)=\lim\limits_{s \to 0} K(s)=
        \lim\limits_{s \to 0} \frac{1}{s^{2}+1}$$

        $$ s_{1}=j, \ s_{2}=-j $$

        Część rzeczywista jest na granicy stabilności, wsp. nie istnieje.

    \end{itemize}


\end{document}
