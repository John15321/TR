\documentclass{article}
% \usepackage{enumitem}

\title{Teoria Regulacji Ćwiczenia, Wtorek 17:05-18:45}
\author{Jan Bronicki 249011 }
\date{}

\begin{document}

\maketitle

\section*{Zadanie 8}

Oryginalna transmitancja:

$$ K(s) = \frac{1}{\left(s + 1\right) \left(s + 2\right)} $$

Podany sygnał:

$$ u(t) = 4sin(\omega t) $$

Transmitancja widmowa:

$$ s = i\omega $$

$$K(i\omega) = \frac{1}{\left(i w + 1\right) \left(i w + 2\right)}$$

Postać sygnału na wyjściu:

$$y(t)\approx\left|K(i\omega)\right|sin\left(\omega t + argK(i \omega)\right)$$



$$ \left|K(i\omega)\right| = \sqrt{\frac{1}{\omega^{2}+1()(\omega^{2}+4)}}\cdot e^{j\cdot 
tan^{-1}\left(\frac{-3\omega}{2-\omega^{2}}\right)} $$

Podstawiam do wzoru:

$$y(t)=4\cdot \sqrt{\frac{1}{754}}\cdot \sin\left(5t+0.577\right)$$


\section*{Zadanie 9}

$$ \lim\limits_{t \to \infty} \lambda(t) = K(0) = \lim\limits_{t \to 0} K(s)  $$

\begin{itemize}
    \item[a)] $\frac{s+2}{(s+3)(s+4)^{2}}$
    
    \item[b)]
    
    \item[c)]
    
    \item[d)]   
\end{itemize}


\end{document}
