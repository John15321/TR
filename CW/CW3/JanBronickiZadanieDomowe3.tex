\documentclass{article}
\usepackage[utf8]{inputenc}
\usepackage{array}
\usepackage{wrapfig}
\usepackage{multirow}
\usepackage{rotating}
\usepackage{tikz}
\usepackage{mathtools}
\usepackage{tabularx}
\usepackage{amsmath}
\usepackage{wrapfig}
\usepackage{mathtools}
\usepackage{graphicx}
\usepackage{ragged2e}
\usepackage{enumitem}

\title{Teoria Regulacji Ćwiczenia, Wtorek 17:05-18:45}
\author{Jan Bronicki 249011 }
\date{}

\begin{document}

\maketitle

\section*{Zadanie 1 z Listy 2}

$$ T y^{\prime}\left(t\right) + y\left(t\right)=k u\left(t\right) $$

\begin{enumerate}[label=\alph*)]

    % u(t) = 0
    \item $u\left(t\right)=0$ 
    \begin{flushleft}

        $ \mathcal{L}\left\{T y^{\prime}\left(t\right) + y\left(t\right) \right\}= \mathcal{L}\left\{0\right\} $

        $ Y\left(s\right)\left[Ts+1\right]-Ty\left(0\right) = 0 $

        $ Y\left(s\right) = y\left(0\right)\frac{1}{s+\frac{1}{T}} $

        $ \mathcal{L}^{-1}\left\{Y\left(s\right)\right\} = \mathcal{L}^{-1}\left\{y\left(0\right)\frac{1}{s+\frac{1}{T}}\right\} $

        $ y\left(t\right) = y\left(0\right)e^{-\frac{1}{T}t} $

    \end{flushleft}




    % u(t) = \delta(t)
    \item $u\left(t\right)=\delta \left(t\right)$
    \begin{flushleft}
        
        $ \mathcal{L}\left\{T y^{\prime}\left(t\right) + y\left(t\right) \right\}= \mathcal{L}\left\{k\right\} $

        $ Y\left(s\right)\left[Ts+1\right]-Ty\left(0\right) = k $

        $ Y\left(s\right) = \left(\frac{k+Ty\left(0\right)}{T}\right)\frac{1}{s+\frac{1}{T}} $

        $ \mathcal{L}^{-1}\left\{Y\left(s\right)\right\} = \mathcal{L}^{-1}\left\{\left(\frac{k+Ty\left(0\right)}{T}\right)\frac{1}{s+\frac{1}{T}}\right\} $

        $ y\left(t\right) = \left(\frac{k+Ty\left(0\right)}{T}\right)e^{-\frac{1}{T}t} $

    \end{flushleft}
    


    % u(t) = 1(t)
    \item $u\left(t\right)=1 \left(t\right)$ 
    
    \begin{flushleft}
        
        $ \mathcal{L}\left\{T y^{\prime}\left(t\right) + y\left(t\right) \right\}= \mathcal{L}\left\{k \cdot1\left(t\right)\right\} $

        $ Y\left(s\right)\left[Ts+1\right]-Ty\left(0\right) = \frac{k}{s} $

        $ Y\left(s\right) = \frac{k+sTy\left(0\right)}{s\left(Ts+1\right)} = \frac{A}{s} + \frac{B}{Ts+1} $

        \[\begin{cases}
            A=k
            \\
            B=T\left(y\left(0\right)-k\right)
        \end{cases}\]

        $  \mathcal{L}^{-1}\left\{Y\left(s\right)\right\} =  \mathcal{L}^{-1}\left\{k\frac{1}{s}+\left(y\left(0\right)-k\right)\frac{1}{s+\frac{1}{T}}\right\} $

        $ y\left(t\right)=k+\left(y\left(0\right)-k\right)e^{-\frac{1}{T}t} $

    \end{flushleft}

    \newpage



    % u(t) = sin(wt)
    \item $u\left(t\right)=sin\left(\omega t\right)$ 
    \begin{flushleft}
        
        $ \mathcal{L}\left\{T y^{\prime}\left(t\right) + y\left(t\right) \right\}= \mathcal{L}\left\{k\cdot sin\left(\omega t\right)\right\} $
        
        $ Y\left(s\right)\left[Ts+1\right]-Ty\left(0\right) = k\frac{\omega}{s^{2}+\omega^{2}} $
        
        $ Y\left(s\right) = \frac{k\omega + Ty\left(0\right)\left(s^{2}+\omega^{2}\right)}{s^{2}+\omega^{2}} = \frac{As+B}{s^{2}+\omega^{2}}+\frac{C}{Ts+1}$
        
        \[\begin{cases}
            A=\frac{-Tk\omega}{T^{2}\omega^{2}+1}
            \\
            B=\frac{k\omega}{T^{2}\omega^{2}+1}
            \\
            C=\frac{T^{2}k\omega}{T^{2}\omega^{2}+1}
        \end{cases}\]
        
        $  \mathcal{L}^{-1}\left\{Y\left(s\right)\right\} =  \mathcal{L}^{-1}\left\{\frac{k}{T^{2}\omega^{2}+1}\left( \frac{-T\omega s}{s^{2}+\omega^{2}}+ 
        \frac{\omega}{s^{2}+\omega^{2}}+
        \frac{T^{2}\omega}{Ts+1}\right)+
        \frac{Ty\left(0\right)}{Ts+1}\right\} $
        
        $ y\left(t\right) = \frac{k}{T^{2}\omega^{2}+1}\left(-T\omega cos\left(\omega t\right)+
        sin\left(\omega t\right)+
        T\omega e^{-\frac{1}{T}t}
        \right)+ y\left(0\right)e^{-\frac{1}{T}t} $
        
    \end{flushleft}
        
        
        
        

    % u(t) = t
    \item $u\left(t\right)= t$ 
    \begin{flushleft}
        
        $ \mathcal{L}\left\{T y^{\prime}\left(t\right) + y\left(t\right) \right\}= \mathcal{L}\left\{k\cdot t\right\} $

        $ Y\left(s\right)\left[Ts+1\right]-Ty\left(0\right) = \frac{k}{s} $
        
        $ Y\left(s\right) = \frac{k+s^{2}Ty\left(0\right)}{s^{2}\left(Ts+1\right)} = \frac{A}{s} + \frac{B}{s^{2}} + \frac{C}{Ts+1} $
        
        \[\begin{cases}
            A=-Tk
            \\
            B=k
            \\
            C=Ty\left(0\right)+T^{2}k
        \end{cases}\]

        $ \mathcal{L}^{-1}\left\{Y\left(s\right)\right\} = \mathcal{L}^{-1}\left\{\frac{-Tk}{s} + \frac{k}{s^{2}} + \frac{y\left(0\right)+Tk}{s+\frac{1}{T}}\right\} $

        $ y\left(t\right) = -Tk+kt+\left(y\left(0\right)+Tk\right)e^{-\frac{1}{T}t} $

    \end{flushleft}





    % u(t) = 2\delta(t) + 1(t)
    \item $u\left(t\right)=2\delta\left(t\right) + 1 \left(t\right)$ 

    \begin{flushleft}

        $ \mathcal{L}\left\{T y^{\prime}\left(t\right) + y\left(t\right) \right\}= \mathcal{L}\left\{k\left(2\delta\left(t\right)+1\left(t\right)\right)\right\} $

        $ Y\left(s\right)\left[Ts+1\right]-Ty\left(0\right) = k\frac{2s+1}{s} $

        $ Y\left(s\right) = k\frac{sTy\left(0\right)+2s+1}{s\left(Ts+1\right)} = \frac{A}{s} + \frac{B}{Ts+1}$

        \[\begin{cases}
            A=k
            \\
            B=2k+Ty\left(0\right)-Tk
        \end{cases}\]

        $ \mathcal{L}^{-1}\left\{Y\left(s\right)\right\} = \mathcal{L}^{-1}\left\{\frac{k}{s}+\frac{2k+Ty\left(0\right)-Tk}{Ts+1}\right\} $

        $ y\left(t\right) = k+\frac{1}{T}\left(2k+Ty\left(0\right)-Tk\right)e^{-\frac{1}{T}t} $

    \end{flushleft}














\end{enumerate}

\end{document}
